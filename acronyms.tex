% \newacronymwithdescr{Label}{Court}{Long}{Description}
\newcommand*{\newacronymwithdescr}[5][]{%
  \newglossaryentry{main-#2}{name={#3},%
  text={(\acs{#2}) #3\glsadd{#2}},%
  description={#5},%
  #1
  }%
  \newacronym{#2}{#3\glsadd{main-#2}}{#4}%
}

% Pas de point final pour les entrées glossaire ou acronymes
\setacronymstyle{sm-short-long}
%\newacronym{IMT}{IMT}{Institut de Mathématiques de Toulouse}
\newacronymwithdescr{IMT}{IMT}{Institut de Mathématiques de Toulouse}{is the main laboratory in mathematics in Toulouse}
\newacronymwithdescr{KSVD}{KSVD}{K Singular Vector Decomposition algorithm}{ia}

%\newglossaryentry{KSVD}{name=K-SVD,description={(K-Singular Vector Decomposition algorithm) is an adaptative method for learning a dictionary using a set of images; each image is decomposed in many small patches that form the atoms of the dictionary. Those atoms are then optimized using OMP (sparse coding of $x$) and a SVD (for the dictionary update of $D$). See \cite{aharon_k-svd:_2006}}}
\newglossaryentry{Adaptative dictionary}{name=,description={}}

% To use the glossary entries:
% \acs{} (for short one)
% \ac{} (for long one on first appearance)

