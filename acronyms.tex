\usepackage{relsize}
\setacronymstyle{short-long}

% Use: \newacronymwithdescr{Label}{Court}{Long}{Description}
\newcommand*{\newacronymwithdescr}[5][]{%
  \newglossaryentry{main-#2}{name={#3},%
  text={(\acs{#2}) #3\glsadd{#2}},%
  description={#5},%
  #1
  }%
  \newacronym{#2}{#3\glsadd{main-#2}}{#4}%
}
% Use: \newglossary{Label}{Name}{Description}
\newcommand*{\newgls}[4][]{%
  \newglossaryentry{#2}{name={#3},%
  description={#4},%
  #1 %
  }%
}




\newacronymwithdescr{IMT}{IMT}{Institut de Mathématiques de Toulouse}{is the main laboratory in mathematics in Toulouse}

\newacronymwithdescr{OMP}{OMP}{Orthogonal Matching Pursuit}{is a greedy algorithm for coding an image into a sparse (with many zeros) code}

\newacronym{PALM}{PALM}{Proximal Alternating Linearized Minimization}

\newacronym[see={PALM}]{PALMTREE}{PALMTREE}{\acl{PALM} for Convolutional Tree Model}

\newacronym{DFT}{DFT}{Discrete Cosine Transform}

\newacronymwithdescr{KSVD}{K-SVD}{K Singular Value Decomposition}{is a generalization of the K-Means algorithm for learning a dictionary on a set of images; each image is decomposed in many small patches that are then learned to form the atoms of the dictionary. See \cite{aharon_k-svd:_2006}}

\newacronymwithdescr[sort={FTL}]{FTL}{\eqref{eq_ftl}}{Fast Transform Learning}{is an optimization problem for dictionary learning. Instead of using the matrix-vector product as in the standard Dictionary Learning \eqref{eq_dl} problem, \eqref{eq_ftl} is based on the \gls{treemodel}}

\newgls{KMeans}{K-Means}{is an unsupervised machine learning algorithm widely used thanks to its easy implementation. It is not specific to image processing and can be applied to any kind of data}

\newgls[text={convolutional tree model}]{treemodel}{Convolutional Tree Model}{is a model of structured dictionary based on convolutions of sparse kernels, proposed in \cite{chabiron_optimization_2016}}


% dictionary update step
% dictionary
% transform
% sparse coding step
% code
% sparse
% adaptative


% To use the glossary entries:
% \acs{} (for short one)
% \ac{} (for long one on first appearance)

