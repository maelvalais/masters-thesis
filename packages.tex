\documentclass[11pt, a4paper,onecolumn, twoside,french]{report}
% openany -> for drafts
\usepackage[a4paper,top=2.5cm,bottom=2.5cm,inner=2.5cm,outer=2.5cm,marginparwidth=2cm,marginparsep=0.05cm]{geometry}
%showframe -> to show the margins and everything
\usepackage[english]{babel}
\usepackage[utf8]{inputenc}
\usepackage[T1]{fontenc}
\usepackage{indentfirst} % alinea on paragraphs
\usepackage[babel=true]{csquotes}
\usepackage{graphicx}
\usepackage[]{amsmath} % fleqn = align blocs on the left
\usepackage{siunitx} % SI units
\usepackage{amssymb} % for convolution sign
%\usepackage{moreverb} % for c, cpp snippets
\usepackage{enumitem} % \begin{enumerate} for more indent in itemize

%\usepackage[english,boxed,lined,onelanguage]{algorithm2e}
%\SetAlCapSkip{1em} % Margin between algo and caption
%\SetAlCapNameSty{textit} % text style for  algo captions
\usepackage{algpseudocode,algorithm,algorithmicx}

\usepackage[toc,page]{appendix}
\usepackage[nottoc,numbib]{tocbibind} % Make the bibliography appear in Summary
\usepackage{fancyhdr} % For titlepage \lhead, \rhead... 
\usepackage[backend=biber,style=ieee-alphabetic,date=long,language=english]{biblatex}
\DeclareFieldFormat*{citetitle}{{\it #1}}
\bibliography{master-thesis.bib}

% To suppress fields in my .bib, try 
% \clearfield, \clearname or \clearlist
% For url date and date, use \DeclareSourcemap
\AtEveryBibitem{\clearfield{doi}}
\AtEveryBibitem{\clearfield{issn}}
\AtEveryBibitem{\clearfield{url}}

\DeclareSourcemap{
  \maps[datatype=bibtex]{
    \map[overwrite=true]{
      \step[fieldset=urldate, null]
    }
  }
}
\usepackage[makeroom]{cancel} % for "crossing" an equation

\usepackage{placeins} % \FloatBarrier for preventing figures to be placed too far away


\def\keywords{\vspace{1em}
{{\it \bf Keywords}:\,\relax%
}}
\def\endkeywords{\par}

%\usepackage{multirow} % Pour colonnes multiples des tableaux
%\usepackage{longtable} % Pour longs tableaux
%\usepackage{array} % Pour \texttt sur tout une colonne
%\usepackage{xcolor} % Pour éviter que footnote ne bug...
%\usepackage{footnote} % Pour les footnotes dans les tableaux
%\makesavenoteenv{tabular} % Pour les footnotes dans les tableaux
%\usepackage{tabularx}
%\usepackage{pdfpages} % Include des pdfs
%\usepackage[nottoc,numbib]{tocbibind} % Pour faire apparaitre la biblio. dans le sommaire
\usepackage{booktabs} 
\usepackage[font={it}]{caption,subcaption}

\usepackage{color} % For \textcolor

\usepackage{mathtools} % for \shortintertext{} in align block

%\usepackage[]{minitoc} % Intermediate 
\usepackage{etoolbox} % For toggle function (to hide titlepage)
\usepackage{nth} % 1st, 3rd, 4th...
\usepackage{bm} % \bm{} for bold font in text and math modes
\usepackage[colorinlistoftodos,prependcaption,textsize=scriptsize]{todonotes} %,textsize=tiny
\usepackage[pdfusetitle]{hyperref} % for linkable refs and title in pdf meta
\usepackage[acronym,toc,shortcuts]{glossaries}
\usepackage[english,capitalise]{cleveref} % capitalise = Fig. instead of fig.

\hypersetup{pdfauthor={Maël Valais, François Malgouyres, Jean-Yves Tourneret, Herwig Wendt},%
            %pdftitle={Your Title},%
            pdfsubject={This master’s thesis presents the algorithm OMP-PALMTREE for optimizing convolutional tree dictionaries from a unique image; inspired from the construction of supports in OMP, this algorithm enhances  PALMTREE by learning the supports (instead of fixing them). The  support adding step, as for OMP, is based on the maximum value of the gradient – which is shown to be giving convincing indications on the best element that should be added. OMP-PALMTREE constitutes the “dictionary update” step in a standard dictionary learning algorithm, and must be associated with a “sparse coding” step (yet to be developed) for it to be used in applications like denoising or image recognition. Thanks to the Convolutional tree model, dictionaries learned using OMP-PALMTREE are faster than any dictionary learned using a standard matrix-vector product  (quadratic computational cost), with a linear computational cost with respect to the size of the image.},%
            pdfkeywords={deep learning, dictionary learning, sparse image representation, non-convex optimization, machine learning, image processing},%
            pdfproducer={LaTeX},%
            pdfcreator={pdfLaTeX}
}


\makeglossary
\makeindex



